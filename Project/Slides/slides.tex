\documentclass[pdf]{beamer}

\usepackage[utf8]{inputenc}
\usepackage[absolute,overlay]{textpos}
\usepackage{adjustbox}
 
\usetheme{Warsaw}


\mode<presentation>{}
%% preamble

\setbeamercolor{framesource}{fg=gray}
\setbeamerfont{framesource}{size=\tiny}

\newcommand{\source}[1]{\begin{textblock*}{4cm}(8.7cm,8.6cm)
    \begin{beamercolorbox}[ht=0.3cm,right]{framesource}
        \usebeamerfont{framesource}\usebeamercolor[fg]{framesource} Source: {#1}
    \end{beamercolorbox}
\end{textblock*}}


\author{Nicolas Drizard}
\date{Advisors: David Bessis, Artem Kozhevnikov, Victor Mazzeo \\ Supervisor: Michalis Vazirgiannis \\ July, $9^{th}$ 2015}
\title{Evaluation of a Neighborhood Weighted Graph between Products \\
$ $   \\
    CONFIDENTIAL}

\begin{document}


%% title frame
\begin{frame}
\titlepage
\end{frame}

\begin{frame}
\frametitle{Outline}
\tableofcontents
\end{frame}

\AtBeginSection[]
{
  \begin{frame}
    \frametitle{Table of Contents}
    \tableofcontents[currentsection]
  \end{frame}
}

\section{Targeting Application}
\subsection{Objective}

\begin{frame}{Objective}
Workflow of the targeting Application:
\begin{enumerate}
\item{Choosing the campaign content}
\item{Choosing the campaign volume}
\item{Learning the model}
\item{Scoring the user base}
\item{Obtaining the target}
\end{enumerate}
\end{frame}

\subsection{Data Environment}

\begin{frame}{Data Tables}
\begin{itemize}
\item{\textbf{user}: socio demographic characteristics and contact data of each user}
\item{\textbf{product}}
\item{\textbf{purchase}}
\item{\textbf{page-view}: information about the navigation of the user on the website, i.e. how he arrived, on which content he clicked...}
\item{\textbf{email}: data about the marketing email sent, in particular if the user opened it or clicked inside}
\end{itemize}
\end{frame}

\begin{frame}{Example}
\begin{figure}
\begin{tabular}{|c|c|}
			\hline
domain &  gmail.com\\ \hline
zipcode &  93400 \\ \hline
user\_id &  420050933 \\ \hline
contactable &  True \\ \hline
firstname & Roger \\ \hline
title & MISTER \\ \hline
Lastname & Dupond \\ \hline
yob & 1978 \\ \hline
first\_purchase\_date & 2001-02-16 \\ \hline
dob & 1978-12-05 \\ \hline
gender & M \\ \hline
country & France \\ \hline
login & bernarddupond \\ \hline
\end{tabular}
\caption{User Table}
\end{figure}
\end{frame}

\begin{frame}{Example}
     \begin{columns}[T] % contents are top vertically aligned
     \begin{column}[T]{4.5cm} % each column can also be its own environment
     \begin{center}
     \begin{figure}
    		\begin{tabular}{|c|c|}
			\hline
product\_id &  188752258\\ \hline
date &  2015-06-30  \\ \hline
user\_id &  420050933 \\ \hline
basket\_id &  391290000 \\ \hline
price & 10.4 \\ \hline
    			\end{tabular}
    			\caption{Purchase Table}
			\end{figure}
		\end{center}
     \end{column}
          \begin{column}[T]{6cm} % alternative top-align that's better for graphics
          \begin{figure}
		\begin{tabular}{|c|c|}
			\hline
product\_id &  156820100\\ \hline
genre &  Livre  \\ \hline
categories\_1 &  Litterature \\ \hline
categories\_2 &  Littérature française \\ \hline
name & un bel morir \\ \hline
price & 5.00 \\ \hline
    			\end{tabular}
    			\caption{Product Table}
			\end{figure}
    	\end{column}
     \end{columns}
\end{frame}

\subsection{Prediction Method}

\begin{frame}{Method}
\begin{figure}
 \includegraphics[scale=0.35]{supervised.png}
\caption{Supervised Learning Workflow}
\source{http://www.astroml.org/sklearn\_tutorial}
\end{figure}
\end{frame}

\begin{frame}{Method}
\begin{itemize}
\item{\textbf{Dimension Reduction}: embeds each category of data in a low dimensional vector space}
\item{\textbf{Regularized Logistic Regression}}
\begin{equation*}
\min_{w}\dfrac{1}{2n_{sample}}||Xw - y||^2 + \alpha ||w||
\end{equation*}
\end{itemize}

\end{frame}

\subsection{Application Demonstration}

\begin{frame}{Targeted Products}
\begin{figure}
 \includegraphics[scale=0.25]{target.png}
\end{figure}
\end{frame}

\begin{frame}{Performance}
\begin{figure}
 \includegraphics[scale=0.25]{gain_curve_ex.png}
\end{figure}
\end{frame}

\begin{frame}{Extract Info}
     \begin{columns}[T] % contents are top vertically aligned
     \begin{column}[T]{5cm} % each column can also be its own 
     \includegraphics[scale=0.25]{sex_distribution.png}
     \end{column}
          \begin{column}[T]{5cm} % alternative top-align that's better for graphics
     \includegraphics[scale=0.25]{age.png}
    	\end{column}
     \end{columns}
\end{frame}

\begin{frame}{Extract Info}
     \begin{columns}[T] % contents are top vertically aligned
     \begin{column}[T]{5cm} % each column can also be its own 
     \includegraphics[scale=0.25]{geographic.png}
     \end{column}
          \begin{column}[T]{5cm} % alternative top-align that's better for graphics
     \includegraphics[scale=0.25]{domain.png}
    	\end{column}
     \end{columns}
\end{frame}

\begin{frame}{Extracts Comparison}
     \begin{columns}[T] % contents are top vertically aligned
     \begin{column}[T]{5cm} % each column can also be its own 
     \includegraphics[scale=0.25]{sex_comp.png}
     \end{column}
          \begin{column}[T]{5cm} % alternative top-align that's better for graphics
     \includegraphics[scale=0.25]{ex_overlap.png}
    	\end{column}
     \end{columns}
\end{frame}

\subsection{Issue}

\begin{frame}{Filter}
\begin{alertblock}{Issue}
Should we apply a filter before the learning or after directly to the scored table?
\end{alertblock}
     \begin{columns}[T] % contents are top vertically aligned
     \begin{column}[T]{4cm} % each column can also be its own 
          \begin{figure}
          \adjustbox{max height=\dimexpr\textheight-3.5cm\relax,
           max width=\textwidth}{
		\begin{tabular}{|c|c|c|}
			\hline
product & in idf & not in idf \\ \hline
theatre & 141 & 71 \\ \hline
orsay & 241 & 227 \\ \hline
veles & 517 & 411 \\ \hline
		\end{tabular}
		}
    			\caption{Original Data}
			\end{figure}
			     \end{column}
          \begin{column}[T]{5.5cm} % alternative top-align that's better for graphics
          \begin{figure}
           \adjustbox{max height=\dimexpr\textheight-5.5cm\relax,
           max width=\textwidth}{
		\begin{tabular}{|c|c|c|c|c|c|}
			\hline
filter & product & robustness & lift 10\% \\ \hline	
pre & orsay & 0.996 & 5.708 \\ \hline
post & orsay & +6\% & +14 \\ \hline
pre & theatre & 0.947 & 4.412 \\ \hline
post & theatre & -7\% & 0 \\ \hline
pre & veles & 0.888 & 5.174 \\ \hline
post & veles & -3\% & +6 \\ \hline
		\end{tabular}
		}
    			\caption{Comparison Results}
			\end{figure}
    	\end{column}
     \end{columns}
\end{frame}

\begin{frame}{Minimum of Positive Events}
\begin{alertblock}{Issue}
How could we target the products with not enough purchase?
\end{alertblock}
\begin{block}{Neighborhood Weighted Graph or Similarity Mapping}
Neighborhood Weighted Graph is a mapping where each structure is mapped to a ranked list of similar products, called buddies. 
\end{block}

\begin{block}{Target Extension}
Extension of the targeted products list with the most similar products to increase the number of positive events considered for the learning until a given threshold.
\end{block}
 
\end{frame}


\section{Quality Evaluation Mission}
\subsection{Metrics}

\begin{frame}{Global Model Performance}
 \begin{columns}[T] % contents are top vertically aligned
     \begin{column}[T]{4cm} % each column can also be its own 
\begin{itemize}
	\item{AUC}
	\item{Robustness}
	\item{Lift}
\end{itemize}
    	\end{column}
    	     \begin{column}[T]{4cm} % each column can also be its own 
			\begin{figure}
	    	     \includegraphics[scale=0.2]{gain_curve.png}
    	     \caption{Gain Curve}		
			\end{figure}			    	     
    	\end{column}
     \end{columns}
\end{frame}

\begin{frame}{Graph Homogeneity \& Stability}
\begin{block}{Homogeneity}
Evaluates the homogeneity of the buddies for any targeted product
	\begin{equation*}
	overlap_{homogeneity} = Extract_{even}^{5\%} \cap Extract_{odd}^{5\%}
	\end{equation*}
\end{block}
\begin{block}{Stability}
Evaluates the stability of the construc- tion of the graph
	\begin{equation*}
	overlap_{stability} = Extract_{train}^{5\%} \cap Extract_{test}^{5\%}
	\end{equation*}
\end{block}
\begin{alertblock}{Remark}
References needed:
$ overlap_{extratag} < ... < overlap_{intratag} $
\end{alertblock}
\end{frame}

\begin{frame}{Buddies Specificity}
\begin{block}{Buddies Specificity Overlap}
Overlap among the buddies lists truncated at a given number (here 50) for different entries weighted by contribution.
\end{block}
			\begin{figure}
	    	     \includegraphics[scale=0.19]{Overlap_Buddies.png}
    	     \caption{Gain Curve}		
			\end{figure}
\end{frame}

\subsection{Demonstration}


\begin{frame}{Aggregated Report}
     \begin{figure}
                \adjustbox{max height=\dimexpr\textheight-5.5cm\relax,
           max width=\textwidth}{
		\begin{tabular}{|c|c|c|c|c|c|c|c|c|}
\hline
auc &  robustness &  $overlap_{intratag}$ & $overlap_{extratag}$ &  $overlap_{homogeneity}$ &  lift\_5 &  lift\_10 \\ \hline
0.69 &  0.91 &  0.93 &  0.77 &  0.93  &  8.90 &  5.99 \\ \hline
\end{tabular}
}
\caption{Model Performance Metrics}
     \end{figure}
          \begin{figure}
              \adjustbox{max height=\dimexpr\textheight-5.5cm\relax,
           max width=\textwidth}{
     		\begin{tabular}{|c|c|c|c|c|c|c|c|c|}
\hline
positive\_events &  length &  entropy & std & max &  argmax &  mean & argmean & argmedian\\ \hline
182 &  150 &  2.99 &  4.61 &  35.4  &  15 &  1.10 & 60 & 50 \\ \hline
\end{tabular}
}
\caption{Detailed Report}
    	\end{figure}
\end{frame}

\begin{frame}{Extract Info}
     \begin{columns}[T] % contents are top vertically aligned
     \begin{column}[T]{5cm} % each column can also be its own 
              \begin{figure}
     \includegraphics[scale=0.25]{curves.png}
     \caption{Gain Curves}
              \end{figure}
     \end{column}
          \begin{column}[T]{5cm} % alternative top-align that's better for graphics
          \begin{figure}
     \includegraphics[scale=0.25]{pressure.png}
          \caption{Pressure Distribution}
              \end{figure}
    	\end{column}
     \end{columns}
\end{frame}

\begin{frame}{Detailed Report}
\begin{figure}
 \includegraphics[scale=0.25]{campaign_overlap.png}
           \caption{Campaing Overlap}
\end{figure}
\end{frame}

\subsection{Perspectives}

\begin{frame}{Applications}
Possible applications of the reporting mission:
\begin{itemize}
\item{Features evaluation}
\item{Benchmark of the positive events minimum required}
\item{Setting dynamically targeting parameters for each client}
\item{Set a trust threshold}
\end{itemize}
\end{frame}

\section{Graph Implementation Methods}
\subsection{Methods}


\begin{frame}{Data}
Let $M_{id}$ be the matrix of purchase occurrence over the user\_id

\vspace{0.6cm}

$ M_{id} $ = \bordermatrix{~  & user_1 & \dots & user_n \cr
                 product_1 & 1 & \dots & 0 \cr
                \vdots & \dots & \dots & \dots \cr
                  product_n & 1 &\dots & 1 \cr} 
                  
\end{frame}


\begin{frame}{Data}
Let $M_{sd}$ be the matrix of purchase occurrence over the socio demo characteristics: ( \textbf{year of birth} (\textbf{yob}), \textbf{sex}, \textbf{firstname} )

\vspace{0.6cm}

$ M_{sd} $ = \bordermatrix{~  & yob_1 & \dots & yob_n & firstname_1 & \dots & firstname_n & M & F \cr
                 product_1 & 1 & \dots & 0 & 1 & \dots & 0 & 1 & 1 \cr
                \vdots & \dots & \dots & \dots & \dots & \dots & \dots & \dots & \dots \cr
                  product_n & 1 &\dots & 1 & 1 & \dots & 1 & 0 & 1\cr} 
                  
\end{frame}

\begin{frame}{Methods}
\begin{itemize}
\item{Jaccard Index}
\item{Singular Value Decomposition (SVD)}
\item{Non Negative Matrix Factorization (NMF)}
\end{itemize}
\end{frame}

\subsection{Comparison}


\begin{frame}{Top Buddies Intersection}
\begin{block}{Remark}
Evaluation of the intersection between the buddies lists from two graphs: \\
x axis : number of products \\
y axis : number of common buddies 
\end{block}
\begin{figure}
\includegraphics[scale=0.20]{jaccard.png}
\caption{Jaccard Index}
\end{figure}
\end{frame}

\begin{frame}{Top Buddies Intersection}
     \begin{columns}[T] % contents are top vertically aligned
     \begin{column}[T]{5cm} % each column can also be its own 
              \begin{figure}
     \includegraphics[scale=0.25]{svd.png}
     \caption{SVD}
              \end{figure}
     \end{column}
          \begin{column}[T]{5cm} % alternative top-align that's better for graphics
          \begin{figure}
     \includegraphics[scale=0.25]{nmf.png}
          \caption{NMF}
              \end{figure}
    	\end{column}
     \end{columns}
\end{frame}

\begin{frame}{Buddies Specificity Overlap: Jaccard}
     \begin{columns}[T] % contents are top vertically aligned
     \begin{column}[T]{5cm} % each column can also be its own 
              \begin{figure}
     \includegraphics[scale=0.25]{jac_id.png}
     \caption{User id}
              \end{figure}
     \end{column}
          \begin{column}[T]{5cm} % alternative top-align that's better for graphics
          \begin{figure}
     \includegraphics[scale=0.25]{jac_sd.png}
          \caption{Socio Demo}
              \end{figure}
    	\end{column}
     \end{columns}
\end{frame}

\begin{frame}{Buddies Specificity Overlap: SVD}
     \begin{columns}[T] % contents are top vertically aligned
     \begin{column}[T]{5cm} % each column can also be its own 
              \begin{figure}
     \includegraphics[scale=0.25]{svd_id.png}
     \caption{User id}
              \end{figure}
     \end{column}
          \begin{column}[T]{5cm} % alternative top-align that's better for graphics
          \begin{figure}
     \includegraphics[scale=0.25]{svd_sd.png}
          \caption{Socio Demo}
              \end{figure}
    	\end{column}
     \end{columns}
\end{frame}

\begin{frame}{Buddies Specificity Overlap: NMF}
     \begin{columns}[T] % contents are top vertically aligned
     \begin{column}[T]{5cm} % each column can also be its own 
              \begin{figure}
     \includegraphics[scale=0.25]{nmf_id.png}
     \caption{User id}
              \end{figure}
     \end{column}
          \begin{column}[T]{5cm} % alternative top-align that's better for graphics
          \begin{figure}
     \includegraphics[scale=0.25]{nmf_sd.png}
          \caption{Socio Demo}
              \end{figure}
    	\end{column}
     \end{columns}
\end{frame}

\subsection{Perspectives}

\begin{frame}{Other Approaches}
Possible applications of the reporting mission:
\begin{itemize}
\item{Distance-based metrics between the distribution of the socio-demo characteristics in the occurrence profile}
\item{Combining Different Methods}
\end{itemize}
\end{frame}

\end{document}